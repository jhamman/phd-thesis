
\chapter{On the relationship between Arctic winter precipitation and minimum sea ice extent}
\label{chap:winter_prec}

This chapter is to be submitted to \textit{Geophysical Research Letters} as

J. Hamman, B. Nijssen., A. Roberts, J. Cassano: On the relationship between Arctic winter precipitation and minimum sea ice extent, \textit{Geophysical Research Letters}.

\section*{Abstract}

Over the past three decades, the Arctic has experienced large declines in summer sea ice cover, permafrost extent and spring snow cover, as well as increases in winter precipitation.
This study explores the relationship between declining Arctic sea ice extent and early winter precipitation across the high-latitude Arctic land masses.
The first part of this paper presents the observed relationship between sea ice extent and winter precipitation. % BN: write it as a paper so there is less change needed later
Using satellite estimates of sea ice extent and precipitation data based on a combination of in-situ observations and global reanalyses, we show that early winter precipitation is negatively correlated with summer sea ice extent and that this relationship is strongest before the year 2000.
After 2000, around the time sea ice extent minima began to decline most rapidly, the relationship between sea ice extent and early winter precipitation degenerates.
This indicates that other processes are driving changes in sea ice extent and winter precipitation.
We hypothesize that the observed correlations between sea ice extent and high winter precipitation are related to anomalous patterns in ocean evaporation and sea ice extent in the fall.
To better understand the physical mechanisms driving the observed changes in the Arctic climate system and the sensitivity of the Arctic climate system to declining sea ice, we have used the fully-coupled Regional Arctic System Model (RASM) to simulate three distinct sea ice climates. % BN: Should this be three rather than two given the next sentence?
The first climate represents normal sea ice extent, while the second and third represent reduced summer sea ice extent.
The second part of this paper analyzes these three RASM simulations, in conjunction with our observation-based analysis, to understand the relationship between poleward moisture transport, sea ice extent, evaporation from the Arctic Ocean, and precipitation.
We will present the RASM-simulated Arctic water budget and demonstrate the role of sea ice extent in driving winter precipitation anomalies.
Finally, we use the Self-Organizing Map (SOM) machine learning technique to identify characteristic patterns of ocean evaporation, sea ice extent, and polar cap convergence that contribute to anomalies in early winter precipitation.

\section{Introduction}
\label{sec:intro_ch5}

In the past three decades, the Arctic region has experienced unprecedented changes in key cryospheric processes.
Rapid declines in sea ice cover have been accompanied by reductions in permafrost extent and spring snow cover, as well as increases in winter precipitation and winter snow accumulations \citep{Kohler_2006,Callaghan_2011,Bulygina_2009}.
These combined changes have had a marked impact on the regional and global climate systems.
Driving much of these changes has been a regional warming trend that is nearly twice as large as the global mean \citep{Serreze_2006c,Screen_2010}.
This disparity in temperature increases is often referred to as Arctic Amplification and is largely explained by the ice-albedo feedback \citep{Curry_1995}.
While this is likely the primary mechanism that leads to rapid warming in the Arctic, other, secondary feedback processes are also at play.
One such feedback process relates the state and fluxes of the Arctic Ocean (sea surface temperatures or SSTs, sea ice cover, evaporation) to precipitation over land, which modulates winter snow cover and permafrost health.

Observational evidence of an amplified hydrologic cycle \citep{Stocker_2005} has been found in the form of increasing precipitation \citep{Rawlins_2006}, runoff \citep{Peterson_2002}, and winter snow accumulations \citep{Kohler_2006,Bulygina_2009}.
While global average precipitation is expected to increase following a response to warming via the Clausius-Clapeyron relationship \citep[e.g. ][]{Held_2006,Stephens_2008,Byrne_2015}, precipitation increases in the Arctic are expected to exceed the global average \citep{Stocker_2005}.
Analysis of the collection Earth system models (ESMs) in the Coupled Model Intercomparison Project Phase 5 \citep[CMIP5; ][]{Taylor_2012} by \citet{Bintanja_2014} indicates that annual precipitation changes in the Arctic may exceed 50\%, with the largest relative increases in the winter over the Arctic Ocean when precipitation has typically been low.
\citet{Bintanja_2014} also identify that most of these changes are due to precipitation sourced from enhanced local evaporation related to retreating sea ice.
This somewhat contradicts previous work that suggested increased poleward moisture transport as the main driver of Arctic precipitation increases. Combined with the large intermodel spread of precipitation changes, this brings into question the sensitivity of modern ESMs. %BN: sensitivity of what to what?
Consequently, the response of precipitation to decreasing sea ice extent remains unclear.
ESMs, along with statistical models, tend to poorly represent the observed decline in summer sea ice extent.
This is evidenced by the intermodel spread among the 39 ESMs analyzed by \citet{Bintanja_2014}.
In their study, changes in sea ice extent between the beginning and end of the twenty-first century ranged between 31-66\%, while changes in precipitation varied by a factor of three to four. % BN: Be careful with the number of significant digits. 31.2 implies that the uncertainty is in the last digit, which is very precise but certainly not very accurate.

Conceptually, a warmer Arctic Ocean with less sea ice will lead to increased surface evaporation and may lead to enhanced divergence of moisture onto land in the form of precipitation.
Because high-latitude land areas are predominantly below freezing in the fall, increases in precipitation during this season are expected to produce deeper snow packs.
This process would then act to insulate the underlying ground during winter and suppress cold season cooling of high-latitude permafrost \citep{Osterkamp_1999,Zhang_2005,Lawrence_2010}.
Further permafrost degradation may be attributed to earlier spring snow melt driven by regional warming and possibly by increased surface infiltration of warm meltwater \citep{Lawrence_2010}.

How the Arctic climate will respond to such large temperature and sea ice changes has been at the forefront of recent studies \citep[e.g. ][]{Kazutoshi_2014,Simmonds_2014,Wegmann_2015,Vihma_2014}.
Here, we investigate the relationship between Arctic winter precipitation, ocean evaporation, and sea ice extent to better understand the terrestrial precipitation response to the ongoing sea ice decline.
Our apriori hypothesis is that the reductions in sea ice extent in would lead to increases in evaporation from the central portions of the Arctic Ocean and precipitation over land during the fall and early winter months. %BN: Are you planning to do formal hypothesis testing? If not then I would rephrase this and stay away from the hypothesis terminology since a reviewer is likely to call you on it.
We test this hypothesis using three simulations spanning a range of sea ice climates from a fully-coupled regional  ESM described in Section \ref{sec:data_models_ch5}.
In Section \ref{sec:results_ch5} we present our analysis of these simulations, first computing the regional freshwater budget following \citet{Serreze_2006a}, then using the Self-Organizing Map (SOM) machine learning technique for dimension reduction and pattern evaluation \citep{Kohonen_1998,Hewitson_2002}.

\section{Data and Methods}
\label{sec:data_models_ch5}

We use three simulations from the Regional Arctic System Model \citep[RASM; ][]{Hamman_2016a,Roberts_2015a}.
RASM is a high-resolution, fully-coupled regional ESM that has been recently developed to improve the representation of coupled Arctic processes.
\citet{Hamman_2016b} provide a complete description of the configuration of RASM for the baseline $RASM_{CONTROL}$ simulation.
Two sensitivity simulations, $RASM_{RSI}$ and $RASM_{RSH}$, representing intermediate and high reductions in sea ice extent are also analyzed.
The configuration of these simulations is identical to $RASM_{CONTROL}$ except in the parameterization of sea ice albedos, which is summarized in Table \ref{table:sims}.
Sea ice albedo is reduced in both simulations to promote later sea ice formation and earlier melt and reduce ice extent. % BN: You need to spend a few more sentences motivating the method that you are using for decreasing sea ice extent (why do it this way rather than prescribing a reduced sea ice extent)
RASM is forced at its lateral boundaries with the ERA-Interim Reanalysis \citep{Dee_2011}.
It is important to note that spectral nudging is applied to temperature and winds in RASM above 500 hPa \citep{Cassano_2016}.
From a practical perspective, this nudging means that the synoptic scale circulation patterns in all three RASM simulations closely match those of ERA-Interim \citep{Glisan_2013}.

% BN: Even though these simulations were described in \citet{Hamman_2016b} the paper needs to stand on its own and you need to provide a few more details: resolution, period (model simulation and analysis), initialization. This can be very short, but is important. Also - did you discard any of the model results at the start of the simulation?

\begin{table}[]
    \centering
    \caption{Summary of RASM simulations used in this chapter.}
    \label{table:sims}
    \begin{tabular}{|l|p{4in}|}
    \hline
    \textbf{Dataset} & \textbf{Sea Ice / Ocean Configuration}                                                                                                         \\ \hline
    $RASM_{CONTROL}$    & Default RASM (see \citet{Hamman_2016b})                                                                                                          \\ \hline
    $RASM_{RSI}$         & \begin{tabular}[c]{@{}l@{}}Ocean: No changes\\ Sea Ice: Snow albedo -0.5 std. dev. of observed.\end{tabular}                                   \\ \hline
    $RASM_{RSH}$         & \begin{tabular}[c]{@{}p{3.5in}}Ocean: No changes\\ Sea Ice: No sea ice initial condition, reduced snow/ice/pond albedo -2.0 std. dev.
    \end{tabular} \\ \hline
    \end{tabular}
\end{table}

Observations of sea ice extent are taken from the National Snow and Ice Data Center (NSIDC) Weekly Sea Ice Extent product \citep{Brodzik_2013}.
Gridded precipitations observations between 1979 and 2014 are taken from CRU TS v.3.23 \citep{Harris_2014}. % BN: Mention spatial and temporal resolution of these data sets
% TODO: reference reanalyses -- % BN: Yes - please include you must have this text from the earlier chapters so just copy it here and shorten it.

\section{Results and Discussion}
\label{sec:results_ch5}
\subsection{Observational evidence}
% Motivation and establishing a connection between sea ice extent and precipitation
Our hypothesized relationship between precipitation and sea ice extent builds on the observed interannual covariance of precipitation and sea ice extent across the central Arctic drainage basin.
Here we define the central Arctic drainage basin as the land areas draining to the core sea ice regions of the Arctic ocean, including the Siberian Shelf (mainly the Kolyma and Lena Rivers) and Canadian coast (mainly the Mackenzie River). % BN: Do you mean covariation rather than covariance?
% BN: A figure would help in describoing the central arctic. You can probably overlay this on Figure 2 or just refer to Figure 4.
Figure \ref{fig:prec_ice_ts} presents the timeseries of observed annual minimum sea ice extent and October-December precipitation in the central Arctic drainage basin from the MERRA and ERA-Interim reanalysis and CRU datasets.
Using the Spearman rank-order correlation measure, the MERRA, ERA, and CRU datasets exhibit correlations of -0.38, -0.40, and -0.52 respectively. % BN: reference for Spearman rank-correlation test
The observed relationship (i.e., negative correlations) is most prominent prior to 2000. % BN: "e.g." means "for example", "i.e." means "that is"
This raises the question whether the two processes are related through a physical coupling that is limited by a threshold mechanism or whether they are perhaps driven by a common forcing.

\begin{figure}
  \centering
  \includegraphics[width=12cm,keepaspectratio]{prec_seaice_ts}
  \caption{Timeseries (1980-2015) of fall precipitation (Oct - Nov) over the central Arctic drainage basin, (left axis) compared to minimum annual sea ice extent (right axis). Note that the right axis is inverted.}
  \label{fig:prec_ice_ts}
\end{figure} % BN: Reduce thickness of sea ice extent line; Figure should follow the text in which it is first mentioned

Figure \ref{fig:prec_spatial_corr} shows the Spearman rank-order correlation coefficients between minimum annual sea ice extent and October-December precipitation from the CRU dataset at each grid cell within the RASM domain.
Across most of the domain, the correlations are found to be below zero, with the most negative correlations occurring across Siberia and North America.
Correlations across Europe are predominantly near zero or positive. % BN: But is Europe part of the central Arctic drainage as you describe it earlier? I think you need to mask these figures to ONLY show the part that is actually used in analysis (the lines in Figure 1).
Generally, a moderately consistent relationship exists between sea ice extent and precipitation, especially in the high-latitude regions of the study domain.
In the aggregate this relationship appears stronger before 2000.
In the following sections, we will investigate possible mechanisms for this relationship.

\begin{figure}
  \centering
  \includegraphics[width=12cm,keepaspectratio]{cru_correlation}
  \caption{Spearman rank-order correlation coefficients between observed minimum annual sea ice extent and gridded Oct-Dec precipitation from CRU. Time period 1980-2015.} % BN: Correlation coefficients do not "compare" things
  \label{fig:prec_spatial_corr}
\end{figure} % BN: Figure should follow the text; Generally don't use titles in figures that are included in manuscripts (that is what captions and axis labels are for)

\subsection{RASM simulations}
% RASM sea ice sensitivity

% BN: This sectin needs to be beefed up. Include some spatial maps of sea icea extents and delve a bit deeper into the results. Do we have minimum sea ice extents at the right time and in the right years (since we are using reanalysis to drive the simulation I assume so.)
Figure \ref{fig:sea_ice_box} shows box-and-whisker plots of sea ice extent from the three RASM simulations escribed in Section \ref{sec:data_models_ch5} compared to the NSIDC observation-based estimates.
Compared to $RASM_{CONTROL}$, $RASM_{RSI}$ and $RASM_{RSH}$ have reductions in mean fall season sea ice cover of 3\% and 11\% respectively. % BN: This is confusing. Why do you switch from the minimum which is used in Figures 1 and 2 to mean seasonal sea ice cover extent
However, compared to the RASM simulations, the observations of sea ice extent demonstrate significantly more interannual variability. % BN: Not only that there distributions look very odd - see comments below
Individual months (e.g. October) have interannual standard deviations that are 1.6 to 3.5 times smaller than the observations.
 % BN: Perhaps, but you need to discuss the odd distribution of the RASM simulations. Whereas the observation are somewhat evenly distributed across the full range of extents, the RASM simulations show an odd distribution, which appears tri-modal. They either cluster near the mean/median or near the extremes. This is true for all three simulations, which seems very odd to me. Are you sure nothing went wrong in the processing. I am very suspicious that there is not a single year that falls in large parts of the distribution of extents. As a reviewer this would be a red flag that something is not right. You need to check this figure with Andrew and include a discussion.

 % BN: Don't use "assert", because you can assert whatever you want but that does not make it so. In this case, I would argue that there is something fishy and you need to provide some more analysis. We cannot just say this is good enough. Representing a map with the mean sea ice extent in each state (or from a simulation that is close to the center of each cluster) may rovide insight into what is going on.

 % BN: Discuss the combination of the three datasets under the SOM discussions. It is not immediately clear to me why you can do that (or what you gain by doing so)

\begin{figure}
  \centering
  \includegraphics[width=12cm,keepaspectratio]{seaice_boxplots}
  \caption{Distribution of fall (Oct - Nov) sea ice extent in RASM and the NSIDC sea ice index. Time period 1985-2015.} % BN: Please change colors. Black dots against dark blue background are difficult to distinguish; Don't use red and green in the same plot; explain in the caption what the individual dots are; explain what the lines on the box plot are (median, interquartile and full range?).
  \label{fig:sea_ice_box}
\end{figure}

% water budget changes
Figure \ref{fig:fwb} presents the freshwater budget for the three RASM simulations, summarized for the fall months (October-December). % BN: Why are you using 1990=2014? I am not sure you explained this before
The total moisture convergence for the three simulations are all within 0.05\% of one another. % BN: Is that because they are driven by the same reanalysis? You named this section "Results and discussion", so you can not just give results and no discussion. Talk about your results. Don't show them to the reader and let them figure it out for themselves.
The reductions in sea ice extent in $RASM_{RSI}$ and $RASM_{RSH}$ are coincident with increases in ocean evaporation of 13\% and 52\%, respectively.
These changes in evaporation from the ocean translate to more modest increases in precipitation over the ocean of 3\% and 11\%.
Associated reductions in convergence over the ocean mask, defined here as P-E between October and December, are found to be 2\% and 10\%. % BN: Is a 2% change meaningful?
Finally, while precipitation over land is found to increase relative to the baseline case for both $RASM_{RSI}$ and $RASM_{RSH}$, the increases are relatively small (1\% and 3\% respectively). % BN: ET over land also increases and is sufficient to account almost entirely for changes in precipitation over land. That is, you don't even need increased ET over the ocean to account for the increased P over land. Process-wise it may be needed, but not from a simple accounting perspective. You need to point that out arther than leaving it as an exercise to the reader
Summarizing the water budget, we find that the forced decreases in $RASM_{RSI}$ and $RASM_{RSH}$ lead to significant increases in ocean evaporation but relatively small changes in precipitation over land, despite relatively large changes in ocean evaporation. % BN: yes - good
Changes in the spatial distribution of precipitation (not shown) lack a regional signal. % BN: From year to year or between simulations?
While this muted response may be partially explained by the limited interannual variability in the RASM sea ice extent, it also indicates that changes in evaporation from the Arctic Ocean will have to be accompanied with changes in circulation patterns if increases in precipitation are to be attributed to sea ice loss.

\begin{figure}
  \centering
  \includegraphics[width=14cm,keepaspectratio]{fresh_water_budget}
  \caption{Arctic freshwater budget, adapted from \citet{Serreze_2006a}.}
  \label{fig:fwb} % BN: Remove title and enlarge figure
  % BN: Be consistent and use the same colors for the model simulations as in the updated figure 3.
  % BN: I am quite sure that the numbers next to the ET_L and (P_L-ET_L-R) are flipped. At least I hope that ET is greater than 47 km3
  % BN: The arrow from land to ocean needs to be labeled with an R
\end{figure}

\subsection{Self-Organizing Maps}
% BN: My main problem with this section is that it reads as if you have found a tool and you are now going to use it just because you have it rather than because you think it may be the best tool for the job. It lacks proper motivation of why you are using SOMs and as a result, the outcomes are rather ambivalent. "I used this technique called SOM and this is what I found". You need to restructure this and provide better motivation for the technique. Rather than being guided by the technique, lead the discussion by what you are interested in (is increased precipitation in the central Arctic a result of a decrease in sea ice extent)? The story is about the latter, not about the technique. Reframing it that way will lead to a better discussion. That starts by removing the "Self-Organizing Maps" as a stand-alone section. It's merely a technique. Focus on the story.

\label{sec:soms}
% SOM analysis
Given the relatively sensitivity in the regional freshwater budget to changes in sea ice extent, we have applied the Self-Organizing Map \citep{Kohonen_1998,Hewitson_2002} technique on standardized evaporation anomalies across the Arctic Ocean. % BN: I don't understand how the second part of the sentence follows from the first part
The goal of this analysis is to provide insight into the coupling between ocean sourced evaporation and anomalies in precipitation over land during the fall season.
SOMs are a useful technique for dimension reduction, allowing for the identification of unique climatological patterns.
They are are a type of unsupervised machine learning, utilizing artificial neural networks. % BN: to do what?
SOMs have been previously applied in studies of polar climatology to study synoptic scale atmospheric circulation \citep[e.g. ][]{Cassano_2007}, extreme weather events \citep[e.g. ][]{Cassano_2015,Glisan_2016}, and coupled ocean-atmosphere processes \citep[e.g. ][]{DuVivier_2016}.
In our analysis, we trained a 2x4 SOM, using standardized ocean evaporation anomalies north of 55$^{\circ}$ N from each of the three RASM simulations described in Section \ref{sec:data_models_ch5} for fall months (October - December) between 1985 and 2014.
In total, the training dataset was comprised of 270 months of evaporation anomalies.
We used a evaluation metric of Euclidean distance, a learning rate of 0.001, and a maximum number of iterations of 2000. % BN: Need tor provide more info on what it means to use a 2x4 SOM and motivate this choice
% BN: Should the training domain be limited to that of Figure 4?
The SOM was initialized with random fields from a standard normal distribution. % BN: This doesn't mean much to people who have not used SOMs. What is the implication of this choice for initialization?
% BN: Are all cells independent?
Further details on the SOM algorithm can be found in \citet{Reusch_2005} or \citet{Cassano_2015}.

% MASTER SOM
The full trained Kohonen Layer (or Master SOM) is shown in Figure \ref{fig:master_som}.
The SOM algorithm identified characteristic spatial patterns of evaporation anomalies.
Here we focus our analysis on four SOM nodes that exhibit the largest hit rate and evaporation patterns of interest: (0,1), (0,3), (1,0), and (1,2). % BN: define hit rate
We will show that the first two SOMs represent generally wet patterns (high precipitation) while the second two represent generally dry patterns (low precipitation). % BN: How will you show that"?
In each of these patterns, the combined position, sign, and magnitude of evaporation anomalies in the central Arctic, Kara/Barents sea, and North Atlantic Ocean are characteristically different.

\begin{figure}
  \centering
  \includegraphics[width=14cm,keepaspectratio]{master_som}
  \caption{Master SOM. The hit frequency is shown in the top left corner.}
  % BN: Needs more descriptive caption
  % BN: Should this be the "hit rate"? Either way, make sure to define it and provide the formula.
  \label{fig:master_som}
\end{figure}

% Composite SOM

Figure \ref{fig:composite_som} presents the composite fields for the four SOM nodes discussed above. % BN: What is a "compaosite field"?
Here we focus on ocean evaporation, sea level pressure (SLP), sea ice concentration, precipitation over land, and snow water equivalent.
For node (0,1), anomalously high evaporation rates from the North Atlantic are accompanied by anomalously low pressure in the region.
This combination is found to lead to positive precipitation anomalies over Northern Europe.
Node (0,3) exhibits positive evaporation anomalies in the Kara and Barents Seas and negative evaporation anomalies in the North Atlantic.
SLPs for this node are anomalously high across North America from Alaska to Greenland, indicating a southward shift of the storm track.
Corresponding precipitation anomalies for node (0,3) are positive over eastern Europe and central Siberia.
Neither of the wet nodes have sea ice concentration anomalies that can be clearly tied to evaporation or precipitation anomalies.  % TODO: we may want to change how the anomalies in sea ice concentration are calculated so that this is more sensitive
% BN: So what do they tell us ...

\begin{figure}
  \centering
  \includegraphics[width=15cm,keepaspectratio]{composite_som}
  \caption{SOM nodes (0,1), (0,3), (1,0), and (1,2) mapped to ocean evaporation, sea level pressure (SLP), sea ice concentration, precipitation over land, and snow water equivalent. Except for sea level pressure, values plotted are the average anomalies across all the members of each node.}
  \label{fig:composite_som}
  % BN: I don't know how to interpret the ET units
  % BN: what is the reference for the anomalies? I don't know how to interpret that the sea ice concentration anomalies are positive for all four SOMs over almost the entire domain.
  % BN: Should this analysis be limited to the Central Arctic?
\end{figure}
% TODO: get the subplots to align better

The selected dry nodes, (1,0) and (1,2), are characterized by negative evaporation anomalies east and north of Greenland, respectively, and positive anomalies in the central Arctic and Siberian Shelf.
Both nodes are mapped to negative sea ice concentration anomalies along the Siberian Shelf.
In node (1,2), positive sea ice concentration anomalies in the Kara Sea correspond to the negative anomalies ocean evaporation, although the spatial extent is considerable larger in the evaporation anomalies.
Node (1,2) also includes positive SLP anomalies across the central Arctic. % BN: Will you discuss SWE? If not, remove it.

% Hit Map
The trained SOM can be used to identify the frequency of occurrence of each pattern as a function of month (October, November, or December) and RASM simulation ($RASM_{CONTROL}$, $RASM_{RSI}$, or $RASM_{RSH}$).
Figure \ref{fig:som_hit_freq} plots the mapping frequency, by month and simulation, for each node in the 2x4 SOM.
Node (0,1), which has a large region of positive evaporation anomalies across the North Atlantic, occurs nearly twice as frequently in $RASM_{RSH}$ as in $RASM_{CONTROL}$.
The opposite is found in node (0,3), where the hit frequency is found to decrease under reduced sea ice conditions.
Interestingly, the hit frequency for dry nodes [e.g. (1,0) and (1,2)] is mostly stable between the simulations.
On the whole, the SOM analysis may indicate that decreasing sea ice extent and the associated evaporation changes are most likely to contribute to wetting trends over land when these changes are accompanied by a shift in atmospheric circulation [e.g. node (0,3)]. % BN: Don't use "may indicate". It's up to you to interpret your results for the reader.
Global climate models tend to suggest that future Arctic climate will be stormier \citep{Vavrus_2012} and that winter circulation patters may trend toward a more variable pattern.
However, the fact that the patterns with large shifts in frequency [(0,1) and (1,2)] shift in directions that limit the impact of sea ice loss, indicates that the precipitation response may be dampened.

\begin{figure}
  \centering
  \includegraphics[width=10cm,keepaspectratio]{som_hit_freq}
  \caption{SOM hit frequency by month and RASM simulation.}
  \label{fig:som_hit_freq}
\end{figure}



\section{Conclusions}
\label{sec:conclusions_ch5}

We have investigated the relationship between sea ice extent and early winter precipitation over the Pan Arctic land masses.
We have demonstrated observed interannual covariance between sea ice extent and precipitation over land, and identified a break in the presumed relations circa year 2000, about the minimum annual sea ice extent began to decline most rapidly. % BN: But I don't think this break eevr comes back in the rest of the narrative which is a real disconnect: "I found a break, but in the rest of the analysis I am going to pretend it was not there"
A water budget analysis highlighted that although relatively large changes in winter evaporation from the Arctic Ocean are expected (50\% in our simulation with the lowest sea ice concentration), these changes are proportionally small when compared to the precipitation flux over land.
Furthermore, the modest increases in terrestrial fall season precipitation simulated by RASM under reduced sea ice conditions indicate that there will have to be significant circulation changes in the future if ocean sourced evaporation is likely to significantly impact precipitation over land.
Our SOM analysis has also pointed us in this direction, which is that the ties between evaporation and precipitation are only robust under certain atmospheric circulations.
Across Eurasia, the dominant coupling between precipitation and Arctic Ocean evaporation requires low pressure anomalies over the central Arctic, similar to the negative phase of the Arctic Oscillation \citep{Thompson_1998}.

Of particular interest is the sea ice state in the Kara and Barents Seas.
As sea ice declines, we expect that this coupling will become stronger, especially if the Arctic is to become stormier.
We have found that sea ice retreat along the Siberian shelf and north Alaskan coast is holds less influence on precipitation regional patterns.
In these areas, the circulation patterns (low pressure in North Pacific) tend to limit the influence of enhanced evaporation along the ice edge.

% BN: Not sure what is going on with the referencing style, but please stick with the same style used in the other sections (and preferably an AGU style). As is, it is writing out the full first names of the authors (if available), which just looks weird (although Donald Knuth was a fervent devotee to this style).
