\chapter{Introduction}
\label{chap:intro}

\section{Background}

The Arctic is a complex and integral part of the global climate system.
It is made up of a diverse landscape that includes open ocean, sea ice, permafrost, ice sheets, tundra, and taiga.
The seasonal cycle of temperature in the Arctic is largely driven by the seasonal cycle in solar radiation, with the highest latitudes receiving no solar radiation in the winter and more than 200 W m$^{-2}$ averaged over the summer months.
Annual average net radiation across the tundra and much of the Arctic Ocean is negative, indicating that the Arctic also acts as a heat sink, balancing poleward heat fluxes from the lower latitudes of the Northern Hemisphere \citep{Serreze_2007}.
The hydrologic cycle in the Arctic is closely tied to the seasonal energy budget.
In the fall and winter, most of the land surface in the region is continuously snow covered.
In the spring and summer, rapid increases in incident radiation lead to ablation of seasonal snow packs and a prominent spring streamflow freshet.

A key motivating factor for studying the Arctic climate system is global climate change and its associated warming.
Global averaged warming caused by increased greenhouse gas emissions is expected to exceed 3 degrees Celsius \citep{IPCC_2014} bu 2100.
The Arctic region is expected to warm at a rate faster than the global average as a result of processes related to ``Polar Amplification'' \citep[e.g.][]{Serreze_2006c,Holland_2003}.
This warming is expected to bring significant changes to the Arctic cryosphere as well as the broader regional and global climate systems.
Significant changes are already being observed, including increasing surface air temperature \citep[e.g.][]{Rigor_2000}, increasing precipitation \citep[e.g.][]{Min_2008}, melting of land ice \citep[e.g.][]{Gardner_2011}, thawing of permafrost \citep[e.g.]{Serreze_2000,Osterkamp_1999}, increasing streamflow and changes in its seasonality \citep[e.g.][]{Dai_2009,McClelland_2006,Peterson_2002,Smith_2007,StJacques_2009}, increasing green vegetation \citep[e.g.][]{Stow_2004,Xu_2013}, increasing wildfires \citep[e.g.][]{Kely_2013}, and decreasing sea ice extent and thickness \citep[e.g.][]{Serreze_2000,Maslowski_2012}.
Taken together, the observational evidence overwhelmingly supports the conclusion that the Arctic is rapidly changing \citep{Serreze_2006b}.
All of these changes are fundamentally linked by the regional water and energy budgets.

Global climate models (GCMs) agree that global temperature will increase in response to increases in atmospheric carbon-dioxide.
However, uncertainty remains regarding the response of individual processes and regional climate systems.
In terms of the representation of key hydroclimatological processes, some GCMs perform quite poorly.
For example, \citet{Alkama_2013} reviewed streamflow statistics from a selection of GCMs in the Coupled Model Intercomparison Project Phase 5 (CMIP5) \citep{Taylor_2012}, and reported that some GCMs have annual regional biases that exceed 50\%.
\citet{Slater_2013} and \citet{Koven_2013} report a wide range of modeled permafrost projections, with many of the global models poorly representing the current extent of continuous permafrost.
Most troubling in the Arctic is the spread of simulated September sea ice extent, in which the range between extreme projections is equal to its current day extent (approximately $8x10^6 km^2$) \citep[e.g.][]{Maslowski_2012}.

Global climate models (GCMs) and the more complex Earth System Models (ESMs) have been widely applied as tools for understanding the global climate system.
While these models tend to agree on global warming trends, they often disagree on individual regional and process responses.
Examples of intermodel disagreement in the Arctic region have been identified in the sensitivity to the polar amplification feedback \citep{Serreze_2006b,Holland_2003}, observed sea ice decline \citep{Stroeve_2007,Zhang_2010}, and the regional response of precipitation to warming and sea ice decline \citep{Bintanja_2014}.
In response to these deficiencies, \citet{Roberts_2010} proposed the development of a high-resolution regional ESM applied over a Pan-Arctic domain.
This model would become the Regional Arctic System Model (RASM) and has been used as the main modeling tool in this dissertation.
Chapter \ref{chap:rasm} further discusses the motivation, development, and application of RASM as a tool for improving our understanding of the coupled Arctic climate system.

\section{Objectives and Research Questions}

Through this dissertation, I aim to better understand the processes and feedbacks that make up the Arctic hydroclimate.
This dissertation focuses on land surface processes that link the water and energy cycles; namely the apportioning of precipitation into runoff and evapotranspiration, the accumulation and ablation of seasonal snow, the terrestrial freshwater flux into the Arctic Ocean, the surface energy balance, and the partitioning of the turbulent heat fluxes (sensible and latent).
I am also interested in the coupled relationships between the terrestrial hydroclimate in the Arctic and the other parts of the Arctic climate system (e.g. ocean, sea ice, biology, etc.).

The primary goal of this dissertation is to develop, evaluate, and apply RASM in order to better understand the Arctic hydroclimate.
My two objectives directing the research in this dissertation are to 1) improve our understanding of the role that coupled land-atmosphere and land-ocean processes play in the Arctic climate system; and 2) develop and improve numerical models that simulate land surface processes in the Arctic.

\clearpage
\begin{mdframed}
  {\bf Research Questions}
  \begin{enumerate}
    \item How well does the Regional Arctic System Model simulate the land surface climate across the pan-Arctic region?
    \item How well does the Regional Arctic System Model capture the freshwater flux into the Arctic Ocean?
    \item What role does the terrestrial freshwater flux play in ocean and sea ice processes?
    \item Do changes in Arctic sea ice extent impact precipitation over the high-latitude land areas in the winter and if so how much?
  \end{enumerate}
\end{mdframed}

\section{Approach}

I address these science questions in three chapters that form the core of this dissertation.
The Chapter \ref{chap:land_surface} \citep[published as ][]{Hamman_2016a} provides an introduction to the baseline land surface climate in the Regional Arctic System Model (version 1.0).
This chapter establishes the strengths and weaknesses of the RASM land surface model and provides the backdrop to the following chapters (Question 1).
In the first part of Chapter \ref{chap:streamflow} \citep{Hamman_2016b}, I introduce the RVIC streamflow routing model, used in RASM to deliver the terrestrial freshwater flux to the ocean model component (Question 2).
In the second part of Chapter \ref{chap:streamflow}, I explore the role the terrestrial freshwater flux plays in the Arctic in terms of salinity and sea ice development (Question 3).
In Chapter \ref{chap:winter_prec} (to be submitted to Geophysical Research Letters), I evaluate the relationship between sea ice extent and winter precipitation (Question 4).
Finally, in \ref{chap:conclusions}, I present conclusions from this collection of research and provide recommendations for future scientific opportunities.
