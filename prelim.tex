%
% ----- copyright and title pages
%
\Title{Understanding the Arctic Hydroclimate Using the Regional Arctic System Model}
\Author{Joseph J. Hamman}
\Year{2016}
\Program{Department of Civil \& Environmental Engineering, University of Washington}

\Chair{Bart Nijssen}{Associate Professor WOT}{Department of Civil \& Environmental Engineering}
\Signature{Erkan Istanbulluoglu}
\Signature{Dennis Lettenmaier}
\Signature{Andrew Roberts}
\Signature{Abigail Swann}

\copyrightpage

\titlepage

\setcounter{page}{-1}
\abstract{
The importance of understanding the Arctic climate system is underscored by the recent and unprecedented observed changes in key climatic processes across the region, and the potential for these changes to impact natural and human activities in coming decades.
Warming associated with global climate change is expected to bring further changes to the Arctic cryosphere as well as the broader regional and global climate systems.
My research has focused on the development and application the Regional Arctic System Model (RASM).
RASM is a fully-coupled regional Earth system model (ESM) applied over a large Pan-Arctic domain.
The development of RASM has been motivated by the need to improve multi-decadal simulations of high-latitude climate and to advance our understanding of the coupled interactions between individual components within the Arctic climate system.
In this dissertation, I present analysis related to the development, evaluation, and application of the components of RASM that simulate land surface processes with the overarching goal of better understanding the Arctic hydroclimate.

This dissertation was made up of three core chapters.
In Chapter \ref{chap:land_surface}, I introduce a novel coupling of the Variable Infiltration Capacity (VIC) model within RASM, evaluating the performance of the VIC compared to observations and other model based datasets.
In Chapter \ref{chap:streamflow}, I present a new river routing scheme (RVIC) for earth system models, again evaluating the model in comparison to in situ observations and model based datasets.
This chapter also presents the development of a new coastal streamflow dataset for ocean modeling applications.
Finally, in Chapter \ref{chap:winter_prec}, RASM was used to evaluate how changes in the sea ice cover in the Arctic Ocean impacted precipitation patterns over land.
}

%
% ----- contents & etc.
%
\tableofcontents
\listoffigures
\listoftables

\chapter*{Glossary}      % starred form omits the `chapter x'
\addcontentsline{toc}{chapter}{Glossary}
\thispagestyle{plain}
%
\begin{glossary}
\item[VIC] Variable Infiltration Capacity model
\item[RASM] Regional Arctic System Model
\item[ESM] Earth System Model
\item[GCM] Global Climate Model
\item[RVIC] ``Route-VIC'' streamflow routing model
\item[AMT] Atmospheric moisture transport
\item[WRF] Weather Research and Forecasting
\item[CICE] Los Alamos Sea Ice model
\item[POP] Parallel Ocean Program model
\item[CESM] Community Earth System Model
\end{glossary}

%
% ----- acknowledgments
%
\acknowledgments{

I would like to thank my academic advisor Dr. Bart Nijssen for his encouragement, support, and guidance.
I am truly thankful for the opportunity to have been a member of his research group.
I would also like to thank the members of my dissertation committee, Dr. Erkan Istanbulluoglu, Dr. Dennis Lettenmaier, Dr. Andrew Roberts, and Dr. Abigail Swann for their insights and encouragement.
Thank you also to the members of the Computational Hydrology Group, it has been a pleasure working alongside all of you.
Much of the research included in this dissertation was made possible by the productive collaboration of the Regional Arctic System Model (RASM) team.
The RASM team was led by Dr. Wieslaw Maslowski (PI), Dr. John Cassano (co-PI), Dr. William Gutowski, Dr. Bart Nijssen (co-PI), and Dr. Andrew Roberts (co-PI).
I am thankful to have been a member of the RASM team and to have had the opportunity to experience such a productive interdisciplinary collaboration.

Thank you to my family for their incredible support throughout my graduate studies - especially to Lauren for her unending encouragement.

My research associated with the RASM project was funded by the U.S. Department of Energy (DOE) Grants DE-FG02-07ER64460 and DE-SC0006856 to the University of Washington.
Supercomputing resources were provided through the Department of Defense (DOD) High Performance Computing Modernization Program at the Army Engineer Research and Development Center and the Air Force Research Laboratory.

At the time of writing, Chapter \ref{chap:rasm} has been published as \citet{Hamman_2016a} in the \textit{Journal of Climate}.
I would like to thank the \textit{Journal of Climate}, published by the American Meteorological Society, for granting the rights to include this article as part of this dissertation.

}
%
% ----- dedication
%
\dedication{\begin{center}For my family.\end{center}}

%
% end of the preliminary pages
